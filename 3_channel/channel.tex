\section{Канал}
Канал - среда передачи данных от передающей антенны к приемной. На пути распространения сигнала в канале могут встречаться препятствия, отражающие либо поглощающие сигнал. Помимо этого, в канале на сигнал накладываются помехи, обусловленные различными факторами. 
\subsection{Функция канала в модели}
В рассматриваемой модели канал представляет собой среду распространения сигнала от передающей антенны к приемной. В данном разделе моделируются помехи и искажения, которые могут оказывать влияние на радиосигнал по мере его распространения: на сигнал накладывается аддитивный шум, моделируется многолучевое распространение, а также вносится частотная рассинхронизация между приемником и источником.
\subsection{Помехи и искажения}
\subsubsection{Шум}
\paragraph{Причина возникновения}
Аддитивный(тепловой) шум - явление, характерное для всех систем связи. В системах связи шум появляется непосредственно на приемнике. Его причиной является колебание ЭДС вследствие теплового движения носителей зарядов в проводниках всех устройств приемника. Более того, усилители в цепи приемника увеличвают мощность неотфильтрованного шума, усиливая его негативное воздействие на сигнал. \newline
Помимо этого, при распространении радиоволны в канале на неё могут накладываться помехи от других радиоэлектронных средств (РЭС), работающих в смежных частотных диапазонах. Такие помехи также являются шумоподобными. \newline 
Наличие шума в канале является причиной возникновения ошибок при демодуляции. Это связано с "размытием" точек созвездия при демодуляции. Пример такого размытия для модуляции 16-QAM показан на рисунке \ref{ch:fig:noisestar}. Величина отклонения принятого значения от переданного определяется отношением сигнал/шум.
\begin{figure} [H]
\centering
\includegraphics[width=0.5\textwidth]{{noised_star}}
\caption{Размытие точек на сигнальном созвездии из-за шума}
\label{ch:fig:noisestar}
\end{figure}
\paragraph{Математика} 
В данной работе при моделировании канала на сигнал накладывается аддитивный белый Гауссовский шум.
\begin{itemize}
\item \textit{Аддитивным} он называется из-за характера воздействия на сигнал: отсчеты шума $N$ суммируются с отсчетами сигнала $X$.
\begin{equation} \label{ch:eq:sigsum}
Y(t) = X(t) + N(t),
\end{equation}
где 
$Y(t)$ - сигнал в зашумленном канале, $X(t)$ - транслируемый сигнал, $N(t)$ - шум.
\item \textit{Белым} - из-за связи со спектром белого света: такой шум имеет равномерную спектральную плотность мощности во всей используемой полосе частот. Спектр белого Гауссовского шума показан на рисунке \ref{ch:fig:noisespec}а. \newline
По той же аналогии шум бывает:
\begin{itemize}
\item \textit{розовым} - низкочастотный шум с зависимостью от частоты порядка $\frac{1}{f}$. Спектр такого шума представлен на рисунке \ref{ch:fig:noisespec}б.
\item \textit{синим} - шум со спектром, зеркальным спектру розового шума. 
\item \textit{серым} - объединение розового и синего шумов. Спектр синего шума показан на рисунке \ref{ch:fig:noisespec}в.
\end{itemize}
\begin{figure} [H]
\centering
\includegraphics[width=\textwidth]{{noise_spec}}
\caption{Спектр а)белого, б)розового и в)синего шумов}
\label{ch:fig:noisespec}
\end{figure}
\item \textit{Гауссовским} - из-за нормального (Гауссовского) распределения временных значений с нулевым средним. Функция нормального распределения показана на рисунке \ref{ch:fig:gauss}.
\begin{figure} [H]
\centering
\includegraphics[width=0.5\textwidth]{{gauss}}
\caption{Распределение Гаусса}
\label{ch:fig:gauss}
\end{figure}
\end{itemize}
 

\subsubsection{Многолучевое распространение}
\paragraph{Причина возникновения}
Многолучевое распространение - явление, возникающее при условии нахождения в точке приема радиосигнала не только прямого, но и отраженных от различных препятствий лучей. Такими препятствиями являются поверхность земли, строения и прочие объекты на пути распространения сигнала (рисунок \ref{ch:fig:multibeam}). 
\begin{figure} [H]
\centering
\includegraphics[width=0.8\textwidth]{{multibeam}}
\caption{Многолучевое распространение}
\label{ch:fig:multibeam}
\end{figure}
Отраженные лучи приходят в точку приема с некоторыми задержками.
\begin{itemize}
\item Если задержка распространения всех лучей мала по сравнению с длительностью канального символа, явление многолучевого распространения приводит только к интерференции лучей, которая, в свою очередь, ведет к \textit{замираниям} сигнала - изменению амплитуды и фазы сигнала, причем  с частотой, намного большей частоты сигнала. Пример изменения спектра сигнала из-за замираний представлен на рисунке \ref{ch:fig:multibeam_spec}.
\begin{figure} [H]
\centering
\includegraphics[width=0.6\textwidth]{{multibeam_spec}}
\caption{Искажение спектра сигнала из-за замираний}
\label{ch:fig:multibeam_spec}
\end{figure}
Методы борьбы с замираниями: разнесенный прием, увеличение мощности сигнала, автоматическая регулировка усиления.
\item Если задержка распространения сравнима с длительностью канального символа, явление многолучевого распространения может привести не только к замираниям, но и к межсимвольной интерференции - частичному наложению канальных символов друг на друга.\newline
Методы борьбы с межсимвольной интерференцией: увеличение защитных интервалов между символами, процедура эквалайзинга на приемнике, увеличение ширины полосы сигнала.
\end{itemize}
По наличию препятствий каналы делятся на 3 типа:
\begin{itemize}
\item Канал \textit{Гаусса} - канал, в котором на приемную антенну приходит только зашуменный сигнал напрямую от передатчика (нет отраженных сигналов). Схематически такой канал представлен на рисунке \ref{ch:fig:channels}а.
\item Канал \textit{Райса} - канал, в котором помимо основного сигнала имеются отраженные от препятствий. На приемной антенне регистрируется интерференция основного и всех отраженных сигналов (рисунок \ref{ch:fig:channels}б).
\item Канал \textit{Рэлея} - канал, в котором на приемной антенне регистрируются только отраженные лучи, а основной (луч прямой видимости) отсутствует из-за препятствий (рисунок \ref{ch:fig:channels}в).
\end{itemize}
\begin{figure} [H]
\centering
\includegraphics[width=0.9\textwidth]{{channels}}
\caption{Типы каналов: а) Гауссовский, б) Райсовский, в) Рэлеевский}
\label{ch:fig:channels}
\end{figure}
\paragraph{Моделирование}
Для защиты от многолучевого распространения в модуляторе используются защитные интервалы. При моделировании данного явления в лабораторной работе вырезается некоторое количество отсчетов сигнала. Таким образом, имитируется временная рассинхронизация приемника и передатчика (см. рисунок \ref{ch:fig:timeshift}).
\begin{figure} [H]
\centering
\includegraphics[width=0.6\textwidth]{{timeshift}}
\caption{Временной сдвиг}
\label{ch:fig:timeshift}
\end{figure}
\subsubsection{Частотный сдвиг и размытие}
\paragraph{Причина возникновения} 
При распространении радиосигнала также могут наблюдаться эффекты частотного сдвига и размытия. Причиной этого явления может служить эффект Доплера - изменение длины волны (а,значит, и частоты) сигнала во время движения передатчика. Также, причиной возникновения данных явлений могут являться сбои в настройке передающего или приемного устройства.
\paragraph{Математика}
В данной работе рассматривается только явление частотного сдвига. Сигнал, распространяющийся от приемника к передатчику в среде, имеет вид:
\begin{equation} \label{ch:eq:signal}
s(t) = \sum_{k} a_k e^{i (\omega_k t + \varphi_k)}
\end{equation}
С математической точки зрения частотный сдвиг влево на $\omega_0$ частотных отсчетов представляет собой домножение исходного сигнала на $e^{i(\omega_0 t)}$:
\begin{equation}
r(t) = s(t) e^{i(\omega_0 t)} = \sum_{k} a_k e^{i ([\omega_k + \omega_0]t + \varphi_k)}
\end{equation}
В общем виде частотный сдвиг, происходящий при распространении сигнала, не является кратным частоте дискретизации сигнала, поэтому имеет смысл отдельно рассмотреть целую и дробную часть частотного сдвига.
\begin{enumerate}
\item При \textit{целочисленном} частотном сдвиге форма спектра сигнала не изменяется, спектр лишь параллельно сдвигается вправо на некоторую частоту, кратную частоте дискретизации ($f=nf_d$), как показано на рисунке \ref{ch:fig:shifterspec}. Для восстановления спектра достаточно перестроить рабочую частоту приемника.
\begin{figure} [H]
\centering
\includegraphics[width=0.75\textwidth]{{shifted_spec}}
\caption{Спектр сигнала с целочисленным частотным сдвигом на 100 отсчетов}
\label{ch:fig:shifterspec}
\end{figure}
\item При \textit{дробном} сдвиге нарушается ортогональность несущих, что приводит к искажению спектра (см. рисунок \ref{ch:fig:frac_shif_spec}). Для восстановления спектра при дробном частотном сдвиге необходимо узнать разность фаз отсчета сигнала и соответствующего ему отсчета защитного интервала. Описанная разность фаз и будет дробным частотным сдвигом.
\begin{figure} [H]
\centering
\includegraphics[width=0.75\textwidth]{{frac_shif_spec}}
\caption{Спектр сигнала, сдвинутого на 100,1 отсчета}
\label{ch:fig:frac_shif_spec}
\end{figure}
\end{enumerate}
\paragraph{Моделирование}
При моделировании частотного сдвига сигнал \eqref{ch:eq:signal}преобразуется следующим образом:
\begin{equation}
r_n(t) = s_n(t) e^{i2\pi (\frac{nk}{N_{FFT}})}
\end{equation}
где $s_n(t)$ и $r_n(t)$ - $n$-ные отсчеты исходного и измененного сигналов соответственно, а $k$ - значение частотного сдвига.
\subsubsection{Канальная характеристика и её учет при демодуляции}
В OFDM-системах часто используется применение канальной характеристики для компенсации вышеописанных искажений. Данный метод основан на выявлении параметров среды распространения сигнала и их учете при демодуляции. \newline
Изменение сигнала при его распространении в канале можно представить как свертку некоторой функции с транслируемым сигналом:
\begin{equation}
\label{ch:eq:character}
r(t) = ch(t)\otimes s(t),
\end{equation}
где \newline
\quad $r(t)$\quad \mbox{--- сигнал на входе приемника,} \newline
\quad $s(t)$\quad \mbox{--- сигнал на ввыходе передатчика,} \newline
\quad $ch(t)$\quad \mbox{--- канальная характеристика.} \newline
Исходя из свойств Преобразования Фурье, спектр сигнала на входе приемника будет равен произведению спектров соответствующих функций:
\begin{equation}
\label{ch:eq:character_spec}
R(\omega) = Ch(\omega)\times S(\omega),
\end{equation}
Таким образом, получение канальной характеристики позволяет восстановить сигнал на приемнике. Для получения канальной характеристики в системах OFDM используют несколько подходов:
\begin{enumerate}
	\item \textbf{Пилотные отсчеты.} В сигнал встраиваются пилотные отсчеты с известными приемнику параметрами, как показано в разделе "Пилоты". %здесь после сбора документа надо вставить ссылку на подраздел Модулятор->Модулятор OFDM->Пилоты
Встраивание пилотных несущих позволяет получать и постоянно обновлять канальную характеристику. Применяется в системах, в которых в любой момент может произойти изменение канальной характеристики, например, при передаче или приеме в движении, либо в системах передачи информации расположенных в городской черте.
	\item \textbf{Пилотные символы.} Во время передачи данных с некоторой заданной периодичностью передаются пилотные символы - символы, полностью известные приемнику. По таким символам приемник получает канальную характеристику и использует ее до появления следующего пилотного символа. В отличие от предыдущего пункта, процесс получения канальной характеристики с применением пилотных символов происходит гораздо быстрее. Такой подход используется при передаче данных в канале, слабо подверженном резкому изменению канальной характеристики. Примером такого канала является вещание на открытой местности.
	\item \textbf{Комбинация пунктов 1 и 2.} Наиболее популярный подход. С помощью использования пилотных символов приемник получает канальную характеристику. С помощью пилотных отсчетов он может подстраиваться под ее изменение в режиме реального времени. Данный подход на данный момент используется практически во всех OFDM-системах.
\end{enumerate}